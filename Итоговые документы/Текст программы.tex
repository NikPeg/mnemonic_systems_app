\documentclass[draft]{article}
\usepackage{cmap}
\usepackage[T1,T2A]{fontenc}
\usepackage[utf8]{inputenc}
\usepackage[russian]{babel}
\usepackage[left=2cm,right=2cm,top=2cm,bottom=2cm,bindingoffset=0cm]{geometry}
\usepackage{tikz}
\usepackage{setspace,amsmath}
\usepackage{titlesec}
\usepackage{lipsum}
\usepackage[usestackEOL]{stackengine}
\usepackage{kantlipsum}
\usepackage{graphicx}
\usepackage{caption}
\usepackage{float}
\usepackage{zref-totpages}
\usepackage{fancyhdr}
\pagestyle{fancy}
\fancyhf{} 
\fancyhead[C]{\thepage\\ RU.17701729.10.03-01 ТЗ 01-1}
\renewcommand{\headrulewidth}{0pt}
\captionsetup[table]{justification=centering}
\usetikzlibrary{positioning}
\graphicspath{{pictures/}}
\DeclareGraphicsExtensions{.pdf,.png,.jpg}
\newcommand\zz[1]{\par{\normalsize\strut #1} \hfill\ignorespaces}
\addto\captionsrussian{\def\refname{}}
\newcommand{\subtitle}[1]{%
  \posttitle{%
    \par\end{center}
    \begin{center}\Large#1\end{center}
   }%
}
\newcommand{\subsubtitle}[1]{%
  \preauthor{%
    \begin{center}
    \large #1 \vskip0.5em
    \begin{tabular}[t]{c}
    }%
}
\begin{document}
\thispagestyle{empty}
\begin{center}
\textbf{
ПРАВИТЕЛЬСТВО РОССИЙСКОЙ ФЕДЕРАЦИИ\\
НАЦИОНАЛЬНЫЙ ИССЛЕДОВАТЕЛЬСКИЙ УНИВЕРСИТЕТ\\
«ВЫСШАЯ ШКОЛА ЭКОНОМИКИ»\\
Факультет компьютерных наук\\
Образовательная программа «Программная инженерия»\\
(ВШЭ ФКН ПИ)}\\
\end{center}
\bigskip
\zz{СОГЛАСОВАНО}УТВЕРЖДАЮ
\zz{Доцент департамента}Академический руководитель
\zz{Программной инженерии,}образовательной программы
\zz{ФКН, к.т.н.}«Программная инженерия»
\zz{\noindent\rule{3cm}{0.4pt} К. Ю. Дегтярёв}профессор департамента программной
\zz{«\noindent\rule{1cm}{0.4pt}»\noindent\rule{2cm}{0.4pt}20\noindent\rule{0.5cm}{0.4pt}г.}инженерии, к.т.н.
\zz{~}\noindent\rule{3cm}{0.4pt} В.В. Шилов
\zz{~}«\noindent\rule{1cm}{0.4pt}»\noindent\rule{2cm}{0.4pt}20\noindent\rule{0.5cm}{0.4pt}г.
\begin{center}
\topskip=0pt
\vspace*{\fill}
\textbf{ПРОГРАММА ДЛЯ ЗАПОМИНАНИЯ ЧИСЛОВЫХ\\
 ДАННЫХ С ИСПОЛЬЗОВАНИЕМ ОСНОВНОЙ\\
 МНЕМОНИЧЕСКОЙ И ДОМИНИКАНСКОЙ СИСТЕМ\\
~\\
~\\
Текст программы\\
~\\
ЛИСТ УТВЕРЖДЕНИЯ\\
~\\
RU.17701729.10.03-01 ТЗ 01-1-ЛУ}\\
\vspace*{\fill}
\end{center}
\zz{~}Исполнитель
\zz{~}Студент группы БПИ204
\zz{~}образовательной программы
\zz{~}«Программная инженерия»
\zz{~}Пеганов Никита Сергеевич
\zz{~}\noindent\rule{3cm}{0.4pt} Н. С. Пеганов
\zz{~}«\noindent\rule{1cm}{0.4pt}»\noindent\rule{2cm}{0.4pt}20\noindent\rule{0.5cm}{0.4pt}г.
\begin{center}
\vspace*{\fill}{
  Москва 2022}
\end{center}
\newpage
\clearpage
\pagenumbering{arabic} 
\begin{textbf}\\
УТВЕРЖДЕН\\
RU.17701729.10.03-01 ТЗ 01-1-ЛУ\\
\end{textbf}
\bigskip
\begin{center}
\topskip=0pt
\vspace*{\fill}
\textbf{ПРОГРАММА ДЛЯ ЗАПОМИНАНИЯ ЧИСЛОВЫХ\\
 ДАННЫХ С ИСПОЛЬЗОВАНИЕМ ОСНОВНОЙ\\
 МНЕМОНИЧЕСКОЙ И ДОМИНИКАНСКОЙ СИСТЕМ\\
~\\
~\\
Текст программы\\
~\\
RU.17701729.10.03-01 ТЗ 01-1-ЛУ}\\
~\\
Листов \ztotpages\\
\vspace*{\fill}
\end{center}
\begin{center}
\vspace*{\fill}{
  Москва 2022}
\end{center}
\newpage
\section{Аннотация}
В данном программном документе представлен текст программы для запоминания числовых данных с использованием основной мнемонической и Доминиканской систем.\\
Оформление данного документа произведено в соответствии с требованиями ЕСПД(ГОСТ 19.102- 77[1], ГОСТ 19.103-77[2], ГОСТ 19.104-78[3], ГОСТ 19.105-78[4], ГОСТ 19.106-78[5], ГОСТ 19.401- 78[6]).
\newpage
\begin{center}
\section {Содержание}
\tableofcontents
\end{center}
\section{Текст программы}
Текст программы выложен в открытый доступ на веб-сервисе github, и может быть найден по ссылке: https://github.com/NikPeg/mnemonic\_systems\_app.
\newpage
\addcontentsline{toc}{section}{Список использованных источников}
\section*{Список использованных источников}
\begin{thebibliography}{}
\bibitem{litlink1} ГОСТ 19.001-77. Единая система программной документации. Термины и определения: утвержден и введен в действие Постановлением Государственного комитета стандартов Совета Министров СССР от 20 мая 1977 г. № 1268 срок введения: с 01.01.1980 г. – URL: https://www.swrit.ru/doc/espd/19.001-77.pdf (дата обращения: 27.01.2023). – Текст: электронный.
\end{thebibliography}
\newpage
\begin{center}
\addcontentsline{toc}{section}{Приложения}
\section*{Приложения}
\end{center}
\zz{}\textbf{Приложение 1\\}
Ссылка на репозиторий проекта с исходным кодом и всеми использованными материалами.\\
https://github.com/NikPeg/mnemonic\_systems\_app\\
\zz{}\textbf{Приложение 2\\}
Ссылка на проект интерфейса в сервисе Figma, отражающий примерную структуру будущего приложения.\\
https://www.figma.com/file/jBcJmt0PREwHvBQRowhaHO/Mnemonic-systems?node-id=38\%3A250\&t=\\
Q8JXDdb3HXM9gGPh-1\\
\end{document}