\documentclass[draft]{article}
\usepackage{cmap}
\usepackage[T1,T2A]{fontenc}
\usepackage[utf8]{inputenc}
\usepackage[russian]{babel}
\usepackage[left=2cm,right=2cm,top=2cm,bottom=2cm,bindingoffset=0cm]{geometry}
\usepackage{tikz}
\usepackage{setspace,amsmath}
\usepackage{titlesec}
\usepackage{lipsum}
\usepackage[usestackEOL]{stackengine}
\usepackage{kantlipsum}
\usepackage{graphicx}
\usepackage{caption}
\usepackage{float}
\usepackage{zref-totpages}
\usepackage{fancyhdr}
\pagestyle{fancy}
\fancyhf{} 
\fancyhead[C]{\thepage\\ RU.17701729.10.03-01 34 01-1}
\renewcommand{\headrulewidth}{0pt}
\captionsetup[table]{justification=centering}
\usetikzlibrary{positioning}
\graphicspath{{pictures/}}
\DeclareGraphicsExtensions{.pdf,.png,.jpg}
\newcommand\zz[1]{\par{\normalsize\strut #1} \hfill\ignorespaces}
\addto\captionsrussian{\def\refname{}}
\newcommand{\subtitle}[1]{%
  \posttitle{%
    \par\end{center}
    \begin{center}\Large#1\end{center}
   }%
}
\newcommand{\subsubtitle}[1]{%
  \preauthor{%
    \begin{center}
    \large #1 \vskip0.5em
    \begin{tabular}[t]{c}
    }%
}
\begin{document}
\thispagestyle{empty}
\begin{center}
\textbf{
ПРАВИТЕЛЬСТВО РОССИЙСКОЙ ФЕДЕРАЦИИ\\
НАЦИОНАЛЬНЫЙ ИССЛЕДОВАТЕЛЬСКИЙ УНИВЕРСИТЕТ\\
«ВЫСШАЯ ШКОЛА ЭКОНОМИКИ»\\
Факультет компьютерных наук\\
Образовательная программа «Программная инженерия»\\
(ВШЭ ФКН ПИ)}\\
\end{center}
\bigskip
\zz{СОГЛАСОВАНО}УТВЕРЖДАЮ
\zz{Доцент департамента}Академический руководитель
\zz{Программной инженерии,}образовательной программы
\zz{ФКН, к.т.н.}«Программная инженерия»
\zz{\noindent\rule{3cm}{0.4pt} К. Ю. Дегтярёв}профессор департамента программной
\zz{«\noindent\rule{1cm}{0.4pt}»\noindent\rule{2cm}{0.4pt}20\noindent\rule{0.5cm}{0.4pt}г.}инженерии, к.т.н.
\zz{~}\noindent\rule{3cm}{0.4pt} В.В. Шилов
\zz{~}«\noindent\rule{1cm}{0.4pt}»\noindent\rule{2cm}{0.4pt}20\noindent\rule{0.5cm}{0.4pt}г.
\begin{center}
\topskip=0pt
\vspace*{\fill}
\textbf{ПРОГРАММА ДЛЯ ЗАПОМИНАНИЯ ЧИСЛОВЫХ\\
 ДАННЫХ С ИСПОЛЬЗОВАНИЕМ ОСНОВНОЙ\\
 МНЕМОНИЧЕСКОЙ И ДОМИНИКАНСКОЙ СИСТЕМ\\
~\\
~\\
Руководство оператора\\
~\\
ЛИСТ УТВЕРЖДЕНИЯ\\
~\\
RU.17701729.10.03-01 34 01-1-ЛУ}\\
\vspace*{\fill}
\end{center}
\zz{~}Исполнитель
\zz{~}Студент группы БПИ204
\zz{~}образовательной программы
\zz{~}«Программная инженерия»
\zz{~}Пеганов Никита Сергеевич
\zz{~}\noindent\rule{3cm}{0.4pt} Н. С. Пеганов
\zz{~}«\noindent\rule{1cm}{0.4pt}»\noindent\rule{2cm}{0.4pt}20\noindent\rule{0.5cm}{0.4pt}г.
\begin{center}
\vspace*{\fill}{
  Москва \the\year{}}
\end{center}
\newpage
\clearpage
\pagenumbering{arabic} 
\begin{textbf}\\
УТВЕРЖДЕН\\
RU.17701729.10.03-01 34 01-1-ЛУ\\
\end{textbf}
\bigskip
\begin{center}
\topskip=0pt
\vspace*{\fill}
\textbf{ПРОГРАММА ДЛЯ ЗАПОМИНАНИЯ ЧИСЛОВЫХ\\
 ДАННЫХ С ИСПОЛЬЗОВАНИЕМ ОСНОВНОЙ\\
 МНЕМОНИЧЕСКОЙ И ДОМИНИКАНСКОЙ СИСТЕМ\\
~\\
~\\
Руководство оператора\\
~\\
RU.17701729.10.03-01 34 01-1-ЛУ}\\
~\\
Листов \ztotpages\\
\vspace*{\fill}
\end{center}
\begin{center}
\vspace*{\fill}{
  Москва \the\year{}}
\end{center}
\newpage
\begin{center}
\section {Содержание}
\tableofcontents
\end{center}
\newpage
\section{Назначение программы}
\subsection{Наименование программы}
\subsubsection{Наименование программы на русском языке}
Программа для запоминания числовых данных с использованием основной мнемонической и Доминиканской систем.
\subsubsection{Наименование программы на английском языке}
A program for storing numerical data using the basic mnemonic and Dominican systems.
\subsection{Функциональное назначение}
Программа представляет из себя удобное приложение для хранения и запоминания чисел, а также изучения основной мнемонической и Доминиканской систем. Приложение не ограничивает пользователя в количестве сохраненных им чисел, ограничивающим фактором является только память телефона.\\
Приложение делится на 5 основных разделов:
\begin{itemize}
\item Запоминаемые числа;
\item Информация об основной мнемонической системе;
\item Справка;
\item Информация о Доминиканской системе;
\item Личный кабинет.
\end{itemize}
Более подробное описание элементов программы представлено в следующем разделе.
\subsection{Эксплуатационное назначение}
Приложение предназначено для пользователей мобильных устройств от 6 лет, сталкивающихся с проблемой запоминания больших чисел. Доступ в интернет не является необходимым для работы программы.
\subsection{Область применения программы}
Программа предназначена для запоминания числовых данных. Она может быть использована как в личных целях, например, для запоминания дней рождений, так и для профессиональных, например, в бухгалтерии. Другие профессиональные сферы в которых может применяться программа:
\begin{itemize}
\item медицина, где врачам и медсестрам необходимо запоминать множество медицинских данных, кодов и номеров пациентов;
\item бухгалтерия, где бухгалтерам необходимо запоминать множество номеров счетов, кодов и других финансовых данных;
\item учеба, где студентам необходимо запоминать большое количество информации, такой как формулы, исторические даты, географические данные и т.д.;
\item наука, где ученым необходимо запоминать множество констант, чисел и других научных данных;
\item бизнес, где менеджерам и предпринимателям необходимо запоминать множество номеров телефонов, адресов, кодов и других контактных данных;
\item другие сферы жизни.
\end{itemize}
\newpage
\section{Условия выполнения программы}
\subsection{Требования к составу и параметрам технических средств}
Программа может быть запущена на мобильном телефоне или планшете с операционной системой Android версии 7.0 и выше. Требования к составу и параметрам технического средства соответствуют требованиям данной операционной системы. Дополнительных ограничений не накладывается.
\subsection{Требования к информационной и программной совместимости}
Для запуска программы она должна быть успешно установлена на мобильное устройство пользователя.
\newpage
\section{Выполнение программы}
\subsection{Установка}
\subsection{Выполнение}
\newpage
\section{Сообщения оператору}
\subsection{Сообщения во время игры}
\subsection{Сообщения во время проверки пользовательской задачи}
\newpage
\addcontentsline{toc}{section}{Список использованных источников}
\section*{Список использованных источников}
\begin{thebibliography}{}
\bibitem{litlink1} ГОСТ 19.001-77. Единая система программной документации. Термины и определения: утвержден и введен в действие Постановлением Государственного комитета стандартов Совета Министров СССР от 20 мая 1977 г. № 1268 срок введения: с 01.01.1980 г. – URL: https://www.swrit.ru/doc/espd/19.001-77.pdf (дата обращения: 27.01.2023). – Текст: электронный.
\end{thebibliography}
\newpage
\begin{center}
\addcontentsline{toc}{section}{Приложения}
\section*{Приложения}
\end{center}
\zz{}\textbf{Приложение 1\\}
Ссылка на репозиторий проекта с исходным кодом и всеми использованными материалами.\\
https://github.com/NikPeg/mnemonic\_systems\_app\\
\zz{}\textbf{Приложение 2\\}
Ссылка на проект интерфейса в сервисе Figma, отражающий примерную структуру будущего приложения.\\
https://www.figma.com/file/jBcJmt0PREwHvBQRowhaHO/Mnemonic-systems?node-id=38\%3A250\&t=\\
Q8JXDdb3HXM9gGPh-1\\
\end{document}