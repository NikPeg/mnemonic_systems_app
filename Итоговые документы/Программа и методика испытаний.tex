\documentclass[draft]{article}
\usepackage{cmap}
\usepackage[T1,T2A]{fontenc}
\usepackage[utf8]{inputenc}
\usepackage[russian]{babel}
\usepackage[left=2cm,right=2cm,top=2cm,bottom=2cm,bindingoffset=0cm]{geometry}
\usepackage{tikz}
\usepackage{setspace,amsmath}
\usepackage{titlesec}
\usepackage{lipsum}
\usepackage[usestackEOL]{stackengine}
\usepackage{kantlipsum}
\usepackage{graphicx}
\usepackage{caption}
\usepackage{float}
\usepackage{zref-totpages}
\usepackage{fancyhdr}
\pagestyle{fancy}
\fancyhf{} 
\fancyhead[C]{\thepage\\ RU.17701729.10.03-01 51 01-1}
\renewcommand{\headrulewidth}{0pt}
\captionsetup[table]{justification=centering}
\usetikzlibrary{positioning}
\graphicspath{{pictures/}}
\DeclareGraphicsExtensions{.pdf,.png,.jpg}
\newcommand\zz[1]{\par{\normalsize\strut #1} \hfill\ignorespaces}
\addto\captionsrussian{\def\refname{}}
\newcommand{\subtitle}[1]{%
  \posttitle{%
    \par\end{center}
    \begin{center}\Large#1\end{center}
   }%
}
\newcommand{\subsubtitle}[1]{%
  \preauthor{%
    \begin{center}
    \large #1 \vskip0.5em
    \begin{tabular}[t]{c}
    }%
}
\begin{document}
\thispagestyle{empty}
\begin{center}
\textbf{
ПРАВИТЕЛЬСТВО РОССИЙСКОЙ ФЕДЕРАЦИИ\\
НАЦИОНАЛЬНЫЙ ИССЛЕДОВАТЕЛЬСКИЙ УНИВЕРСИТЕТ\\
«ВЫСШАЯ ШКОЛА ЭКОНОМИКИ»\\
Факультет компьютерных наук\\
Образовательная программа «Программная инженерия»\\
(ВШЭ ФКН ПИ)}\\
\end{center}
\bigskip
\zz{СОГЛАСОВАНО}УТВЕРЖДАЮ
\zz{Доцент департамента}Академический руководитель
\zz{Программной инженерии,}образовательной программы
\zz{ФКН, к.т.н.}«Программная инженерия»
\zz{\noindent\rule{3cm}{0.4pt} К. Ю. Дегтярёв}профессор департамента программной
\zz{«\noindent\rule{1cm}{0.4pt}»\noindent\rule{2cm}{0.4pt}20\noindent\rule{0.5cm}{0.4pt}г.}инженерии, к.т.н.
\zz{~}\noindent\rule{3cm}{0.4pt} В.В. Шилов
\zz{~}«\noindent\rule{1cm}{0.4pt}»\noindent\rule{2cm}{0.4pt}20\noindent\rule{0.5cm}{0.4pt}г.
\begin{center}
\topskip=0pt
\vspace*{\fill}
\textbf{ПРОГРАММА ДЛЯ ЗАПОМИНАНИЯ ЧИСЛОВЫХ\\
 ДАННЫХ С ИСПОЛЬЗОВАНИЕМ ОСНОВНОЙ\\
 МНЕМОНИЧЕСКОЙ И ДОМИНИКАНСКОЙ СИСТЕМ\\
~\\
~\\
Программа и методика испытаний\\
~\\
ЛИСТ УТВЕРЖДЕНИЯ\\
~\\
RU.17701729.10.03-01 51 01-1-ЛУ}\\
\vspace*{\fill}
\end{center}
\zz{~}Исполнитель
\zz{~}Студент группы БПИ204
\zz{~}образовательной программы
\zz{~}«Программная инженерия»
\zz{~}Пеганов Никита Сергеевич
\zz{~}\noindent\rule{3cm}{0.4pt} Н. С. Пеганов
\zz{~}«\noindent\rule{1cm}{0.4pt}»\noindent\rule{2cm}{0.4pt}20\noindent\rule{0.5cm}{0.4pt}г.
\begin{center}
\vspace*{\fill}{
  Москва \the\year{}}
\end{center}
\newpage
\clearpage
\pagenumbering{arabic} 
\begin{textbf}\\
УТВЕРЖДЕН\\
RU.17701729.10.03-01 51 01-1-ЛУ\\
\end{textbf}
\bigskip
\begin{center}
\topskip=0pt
\vspace*{\fill}
\textbf{ПРОГРАММА ДЛЯ ЗАПОМИНАНИЯ ЧИСЛОВЫХ\\
 ДАННЫХ С ИСПОЛЬЗОВАНИЕМ ОСНОВНОЙ\\
 МНЕМОНИЧЕСКОЙ И ДОМИНИКАНСКОЙ СИСТЕМ\\
~\\
~\\
Программа и методика испытаний\\
~\\
RU.17701729.10.03-01 51 01-1-ЛУ}\\
~\\
Листов \ztotpages\\
\vspace*{\fill}
\end{center}
\begin{center}
\vspace*{\fill}{
  Москва \the\year{}}
\end{center}
\newpage
\begin{center}
\tableofcontents
\end{center}
\newpage
\section {Объект испытаний}
Мобильное приложение, позволяющее русскоговорящему пользователю применять основную мнемоническую и Доминиканскую системы для запоминания больших чисел с помощью смартфона. Для этого реализованы:
\begin{itemize}
    \item Добавление запоминаемых чисел;
    \item Сохранение чисел в памяти;
    \item Запоминание ассоциаций между числами и буквами;
    \item Справочная информация о мнемонических системах
\end{itemize}
Краткое наименование программы: программа для запоминания числовых данных с использованием основной мнемонической и доминиканской систем\\
\newpage
\section {Цель испытаний}
Целью испытаний программы для запоминания числовых данных с использованием основной мнемонической и доминиканской систем является проверка того, что программа верно реализует функционал, описанный в разделе "{}Требования к программе"{} технического задания.
\newpage
\section {Требования к программе или программному изделию}
\subsection{Краткое описание приложения}
Приложение должно позволить русскоговорящим пользователям применять основную мнемоническую и Доминиканскую системы для запоминания больших чисел. Для этого должна быть реализована возможность выбора языка, используемого приложением. Создана справка, позволяющая узнать всю нужную информацию об обеих системах. Помимо этого, необходимо использовать карточки для выучивания связи между цифрами и буквами в обеих системах. Кроме того, будет реализована работа со словарем: пользователь сможет открывать используемый в приложении словарь и добавлять туда новые слова.
\subsection{Требования к функциональным характеристикам}
\begin{itemize}
\item Должна быть реализована возможность сохранять запоминаемые числа;
\item Должен быть реализован функционал подбора слов и имён для мнемонических систем;
\item Должна быть предоставлена возможность узнать подробную информацию об обеих системах;
\item Необходимо позволить пользователю редактировать используемый словарь и добавлять новые слова;
\item Должен быть обеспечен выбор из нескольких языков.
\end{itemize}
\subsection{Требования к интерфейсу}
Приложение состоит из 11 экранов (результат предварительного проектирования по состоянию на конец марта 2023 года):
\begin{itemize}
\item \textbf{Запоминаемые числа}
\item \textbf{Добавление числа}
\item \textbf{Поиск слов для запоминания}
\item \textbf{Редактирование числа}
\item \textbf{Карточки с числами в основной мнемонической системе}
\item \textbf{Справка с информацией об обеих системах}
\item \textbf{Карточки с числами в Доминиканской системе}
\item \textbf{Панель настроек}
\item \textbf{Список слов в словаре}
\item \textbf{Добавление слова в словарь}
\item \textbf{Выбор языка}
\end{itemize}
Примерный вид описанных экранов может быть увиден в прототипе интерфейса, созданном в приложении Figma. См. приложение №2.\\
~\\
\textbf{Требование к интерфейсу}\\
Интерфейс должен быть оформлен в соответствии с дизайн-системой Material Design.\\
Более подробное описание требований к интерфейсу описано на странице 14 Технического Задания, входящего в пакет документации, предоставляемой в рамках курсовой работы.
\subsection{Разрешения}
В данном приложении у пользователя спрашивается только одно разрешение — доступ к файловой системе. Оно должно впервые запрашиваться у пользователя при попытке добавить фотографию к числу.
\subsection{Требования к входным данным}
Определенных требований к входным данным не предусмотрено. Размер введённых пользователем чисел и их количество ограничены только памятью мобильного устройства. Однако числа должны вводиться арабскими числами в десятичной системе, без пробелов или невидимых символов.
\subsection{Требования к выходным данным}
\begin{itemize}
\item На экране добавления нового числа должны появляться слова из словаря пользователя (при выборе основной мнемонической системы) или инициалы (при выборе Доминиканской системы). При этом слова и инициалы должны соответствовать выбранному пользователем языку.
\item На экранах мнемонических систем должны появляться буквы, соответствующие цифре, изображенной на карточке. При этом буквы должны соответствовать выбранному пользователем языку.
\item На справочном экране должна быть представлена подробная информация об обеих мнемонических системах и справка об использовании приложения. Язык справочной информации должен соответствовать выбранному пользователем в настройках.
\end{itemize}
\subsection{Требования к надёжности}
Приложение не должно аварийно завершаться в процессе работы ни при каких обстоятельствах. Программа не должна допускать ввода нечисловых данных в числовые поля.
\subsection{Условия эксплуатации}
Условия эксплуатации программы совпадают с условиями эксплуатации устройства, на котором она запущена. Дополнительных условий не накладывается.
\subsection{Требования к составу и параметрам технических средств}
Программа может быть запущена на мобильном телефоне или планшете с операционной системой Android версии 7.0 и выше. Требования к составу и параметрам технического средства соответствуют требованиям данной операционной системы. Дополнительных ограничений не накладывается.
\subsection{Требования к информационной и программной совместимости}
Программа должна быть написана на языке программирования Java SE в среде разработки Android Studio. Итоговый результат программы — скомпилированный apk-файл.
\subsection{Требования к маркировке и упаковке}
Специальных требований к маркировке и упаковке не накладывается.
\subsection{Требования к транспортированию и хранению}
Устройство может распространяться на физических устройствах, например, жёстких дисках, флеш-накопителях, DVD-дисках. В таком случае требования к транспортированию и хранению совпадают с требованиями данных устройств.
\subsection{Дальнейшая работа}
Область применения обеих изучаемых в работе мнемонических систем обширна, поэтому приложение имеет множество возможностей для расширения. Дополнительные функции могут быть реализованы автором в том случае, если в срок будет реализован основной функционал.\\
В краткосрочной перспективе могут быть добавлены:
\begin{itemize}
\item Статистика, позволяющая пользователю отслеживать свои результаты;
\item Возможность делиться сохраненными числами с другими пользователями;
\item Отслеживание статистики друзей для создания соревновательного эффекта;
\end{itemize}
В случае востребованности приложения пользователями, появится смысл выходить на новые рынки. Для этого потребуется добавление нового функционала, а также перевода интерфейса приложения на другие языки.\\
В долгосрочной перспективе могут быть добавлены: 
\begin{itemize}
\item Поддержка других языков, кроме русского и английского;
\item Другие мнемонические системы, например, система Катапаяди;
\item Способы тренировки памяти для её улучшения.
\end{itemize}
\newpage
\section {Требования к программной документации}
\subsection{Состав программной документации}
\begin{itemize}
\item «Программа для запоминания числовых данных с использованием основной мнемонической и Доминиканской систем». Техническое задание (ГОСТ 19.20178);
\item «Программа для запоминания числовых данных с использованием основной мнемонической и Доминиканской систем». Пояснительная записка (ГОСТ 19.40479);
\item «Программа для запоминания числовых данных с использованием основной мнемонической и Доминиканской систем». Руководство оператора (ГОСТ 19.50579);
\item «Программа для запоминания числовых данных с использованием основной мнемонической и Доминиканской систем». Программа и методика испытаний (ГОСТ 19.30178);
\item «Программа для запоминания числовых данных с использованием основной мнемонической и Доминиканской систем». Текст программы (ГОСТ 19.40178).
\end{itemize}
\subsection{Специальные требования к программной документации}
Программная документация подготовлена в соответствии с требованиями к программным проектам студентов образовательной программы "{}Программная инженерия"{}.\\
~\\
При составлении документации использовался международный стандарт для подготовки технического описания программы IEEE Std 1016-1998 «IEEE Recommended Practice for Software Design Descriptions» \cite{litlink1}, а также ГОСТ 19 Единая система программной документации (ЕСПД) \cite{litlink2}.
\newpage
\section {Средства и порядок испытаний}
\subsection {Технические средства}
Мобильный телефон Xiomi Mi9. Характеристики:
\begin{itemize}
\item операционная система Android 10;
\item 4 ГБ свободной памяти;
\item 6 ГБ оперативной памяти;
\item процессор 8 яндер Max 2.84 ГГц;
\end{itemize}
\subsection {Программные средства}
Эмуляторы, встроенные в IDE Android Studio:
\begin{itemize}
\item Nexus 3-6;
\item Pixel 2-6;
\item Galaxy Nexus;
\item и другие виртуальные устройства.
\end{itemize}
\subsection {Порядок проведения испытаний}
Испытания проводятся в следующем порядке:
\begin{enumerate}
\item  Проверка соответствия требованиям к программной документации;
\item  Проверка соответствия требованиям к интерфейсу;
\item Проверка соответствия требованиям к функциональным характеристикам;
\item Проверка соответствия требованиям к надежности.
\end{enumerate}
\newpage
\section {Методы испытаний}
\subsection{Установка приложения}
Программа для запоминания числовых данных с использованием основной мнемонической и Доминиканской систем распространяется в виде APK-файла. Для установки программы требуется запустить файл с мобильного устройства. При запуске открывается список чисел, добавленных пользователем.
\subsection{Исполнение выполнения требований к программной документации}
Состав программной документации проверяется вручную. Список обязательных документов, представленный в системе ЛМС:
\begin{itemize}
\item Техническое задание (если проект командный, то общее + индивидуальное)
\item Руководство оператора или Руководство программиста 
\item Программа и методика испытаний
\item Текст программы
\end{itemize}
Количество документов и их внутреннее содержание соответствует требуемому списку.\\
~\\
Проверка на соответствие докмуентации ГОСТ также происходит вручную. Все документы удовлетворяют заявленным требованиям.
\subsection{Исполнение выполнения требований к интерфейсу}
При запуске приложения открывается экран — список чисел, добавленный ранее пользователем. Список представлен в виде плиток с картинками, названием числа и самим числом. При нажатии на одну из плиток открывается экран редактирования числа, на котором представлены:
\begin{itemize}
\item название числа;
\item само число;
\item картинка, ассоциирующаяся с числом;
\item выбранная пользователем мнемоническая система;
\item список слов, выбранных пользователем для запоминания числа. 
\end{itemize}
~\\
В правом нижнем углу экрана находится кнопка "{}Редактировать число"{}. При нажатии на неё открывается аналогичный экран, однако все поля могут быть отредактированы пользователем. В верхней части экрана находится плашка, на которой есть кнопка "{}Назад"{}. При нажатии на неё происходит возврат к предыдущему экрану.\\
~\\
На экране со списком чисел находится кнопка "{}Добавить число"{}. При нажатии на неё открывается экран, на котором можно ввести название числа, само число, выбрать картинку, соответсвующую числу. Также есть выпадающий список, позволяющий выбрать мнемоническую систему. В нижней части экрана находится кнопка "{}Далее"{}.\\
~\\
При нажатии на неё открывается экран выбора слов для использования выбранной мнемонической системы. В верхней части экрана указано пояснение: "Выберете цифры для поиска слов:". Под ним отражены
цифры указанного пользователем числа в виде кнопок. При нажатии поочередно каждой из этих кнопок,
под ними появляется список слов, соответствующих выбранным цифрам в указанной мнемонической
системе. При нажатие любого из этих слов, они добавляются в поле ввода в нижней части экрана. Под
этим полем ввода присутствует кнопка "Сохранить число".\\
~\\
В нижней части экрана присутствует "{}Подвал"{}. На нём есть две кнопки, означающие основную мнемоническую и Доминиканскую систему. При нажатии на них открываются "{}карточки"{}, позволяющие пользователю выучить ассоциации цифр и букв.\\
~\\
В центре "{}подвала"{} расположена кнопка, открывающая экран информации. На нём представлено подробное описание основной мнемонической и Доминиканской систем.\\
~\\
В правой части нижней панели расположена кнопка настроек. При нажатии на неё открывается экран, на котором может быть выбран язык приложения, а также открыт словарь, с используемыми в приложении словами.\\
~\\
При нажатии на кнопку выбора языка открывается экран выбора языка. Пока присутствуют два языка: русский и английский.\\
~\\
При нажатии на кнопку "{}Словарь"{} открывается экран словаря, в котором присутствует поиск слов, под ним — список всех слов словаря в алфавитном
порядке. В правой нижней части экрана присутствует кнопка "Добавить", позволяющая добавить слово в словарь.\\
~\\
При нажатии кнопки "{}Добавить слово"{} открывается всплывающее окно, в котором можно ввести слова и сохранить его.\\
~\\
Таким образом, в приложении присутствуют 11 экранов, описанных в техническом задании.
\subsection{Исполнение выполнения требований к функциональным характеристикам}
\subsubsection{Возможность сохранять запоминаемые числа}
Такая возможность действительно присутствует, так как запоминаемые числа хранятся на главном экране в виде плиток.
\subsubsection{Функционал подбора слов и имён для мнемонических систем}
Такой функционал присутствует при выборе одной из мнемонических систем во время добавления числа.
\subsubsection{Возможность узнать подробную информацию об обеих системах}
Подробная информация о мнемонических системах представлена на экране с информацией.
\subsubsection{Возможность редактировать используемый словарь и добавлять новые слова}
Используемый словарь действительно открывается из экрана настроек, и в нём есть возможность добавлять новые числа.
\subsubsection{Выбор из нескольких языков}
На экране настроек присутствует кнопка, открывающая выбор из нескольких языков.\\
~\\
Таким образом, приложение выполняет все требования к функциональным характеристикам.
\subsection{Исполнение требований к надежности}
\subsubsection{Отсутствие аварийного завершения при любых действиях пользователя}
Во время тестирования приложения не обнаружено аварийного завершения при любых действиях.
\subsubsection{Невозможность ввода некорректных данных в окно проверки}
Поля ввода данных находятся только на экранах добавления числа и их редактирования. При вводе букв или некорректных символов в поле ввода числа кнопка "{}Далее"{} не становится активной, что не позволяет пользователю добавить некорректное число. Остальные поля позволяют ввести любые символы, пока их количество не превышает заявленную длину.\\
~\\
Таким образом, программа соответсвует всем заявленным требованиям к надёжности.
\newpage
\addcontentsline{toc}{section}{Список использованных источников}
\section*{Список использованных источников}
\begin{thebibliography}{}
\bibitem{litlink1} \textit{IEEE} (1998) IEEE Recommended Practice for Software Design Descriptions // Сайт ieeexplore.ieee.org. 4 декабря (https://ieeexplore.ieee.org/document/741934)
\bibitem{litlink2} ГОСТ 19.001-77. Единая система программной документации. Термины и определения: утвержден и введен в действие Постановлением Государственного комитета стандартов Совета Министров СССР от 20 мая 1977 г. № 1268 срок введения: с 01.01.1980 г. – URL: https://www.swrit.ru/doc/espd/19.001-77.pdf (дата обращения: 27.01.2023). – Текст: электронный.
\end{thebibliography}
\newpage
\begin{center}
\addcontentsline{toc}{section}{Приложения}
\section*{Приложения}
\end{center}
\zz{}\textbf{Приложение 1\\}
Ссылка на репозиторий проекта с исходным кодом и всеми использованными материалами.\\
https://github.com/NikPeg/mnemonic\_systems\_app\\
\zz{}\textbf{Приложение 2\\}
Ссылка на проект интерфейса в сервисе Figma, отражающий примерную структуру будущего приложения.\\
https://www.figma.com/file/jBcJmt0PREwHvBQRowhaHO/Mnemonic-systems?node-id=38\%3A250\&t=\\
Q8JXDdb3HXM9gGPh-1\\
\newpage
\section*{ЛИСТ РЕГИСТРАЦИИ ИЗМЕНЕНИЙ}
\end{document}