\documentclass[draft]{article}
\usepackage{cmap}
\usepackage[T1,T2A]{fontenc}
\usepackage[utf8]{inputenc}
\usepackage[russian]{babel}
\usepackage[left=2cm,right=2cm,top=2cm,bottom=2cm,bindingoffset=0cm]{geometry}
\usepackage{tikz}
\usepackage{setspace,amsmath}
\usepackage{titlesec}
\usepackage{lipsum}
\usepackage[usestackEOL]{stackengine}
\usepackage{kantlipsum}
\usepackage{graphicx}
\usepackage{caption}
\usepackage{float}
\usepackage{zref-totpages}
\usepackage{fancyhdr}
\pagestyle{fancy}
\fancyhf{} 
\fancyhead[C]{\thepage\\ RU.17701729.10.03-01 ТЗ 01-1}
\renewcommand{\headrulewidth}{0pt}
\captionsetup[table]{justification=centering}
\usetikzlibrary{positioning}
\graphicspath{{pictures/}}
\DeclareGraphicsExtensions{.pdf,.png,.jpg}
\newcommand\zz[1]{\par{\normalsize\strut #1} \hfill\ignorespaces}
\addto\captionsrussian{\def\refname{}}
\newcommand{\subtitle}[1]{%
  \posttitle{%
    \par\end{center}
    \begin{center}\Large#1\end{center}
   }%
}
\newcommand{\subsubtitle}[1]{%
  \preauthor{%
    \begin{center}
    \large #1 \vskip0.5em
    \begin{tabular}[t]{c}
    }%
}
\begin{document}
\thispagestyle{empty}
\begin{center}
\textbf{
ПРАВИТЕЛЬСТВО РОССИЙСКОЙ ФЕДЕРАЦИИ\\
НАЦИОНАЛЬНЫЙ ИССЛЕДОВАТЕЛЬСКИЙ УНИВЕРСИТЕТ\\
«ВЫСШАЯ ШКОЛА ЭКОНОМИКИ»\\
Факультет компьютерных наук\\
Образовательная программа «Программная инженерия»\\
(ВШЭ ФКН ПИ)}\\
\end{center}
\bigskip
\zz{СОГЛАСОВАНО}УТВЕРЖДАЮ
\zz{Доцент департамента}Академический руководитель
\zz{Программной инженерии,}образовательной программы
\zz{ФКН, к.т.н.}«Программная инженерия»
\zz{\noindent\rule{3cm}{0.4pt} К. Ю. Дегтярёв}профессор департамента программной
\zz{«\noindent\rule{1cm}{0.4pt}»\noindent\rule{2cm}{0.4pt}20\noindent\rule{0.5cm}{0.4pt}г.}инженерии, к.т.н.
\zz{~}\noindent\rule{3cm}{0.4pt} В.В. Шилов
\zz{~}«\noindent\rule{1cm}{0.4pt}»\noindent\rule{2cm}{0.4pt}20\noindent\rule{0.5cm}{0.4pt}г.
\begin{center}
\topskip=0pt
\vspace*{\fill}
\textbf{ПРОГРАММА ДЛЯ ЗАПОМИНАНИЯ ЧИСЛОВЫХ\\
 ДАННЫХ С ИСПОЛЬЗОВАНИЕМ ОСНОВНОЙ\\
 МНЕМОНИЧЕСКОЙ И ДОМИНИКАНСКОЙ СИСТЕМ\\
~\\
~\\
Пояснительная записка\\
~\\
ЛИСТ УТВЕРЖДЕНИЯ\\
~\\
RU.17701729.10.03-01 ТЗ 01-1-ЛУ}\\
\vspace*{\fill}
\end{center}
\zz{~}Исполнитель
\zz{~}Студент группы БПИ204
\zz{~}образовательной программы
\zz{~}«Программная инженерия»
\zz{~}Пеганов Никита Сергеевич
\zz{~}\noindent\rule{3cm}{0.4pt} Н. С. Пеганов
\zz{~}«\noindent\rule{1cm}{0.4pt}»\noindent\rule{2cm}{0.4pt}20\noindent\rule{0.5cm}{0.4pt}г.
\begin{center}
\vspace*{\fill}{
  Москва 2022}
\end{center}
\newpage
\clearpage
\pagenumbering{arabic} 
\begin{textbf}\\
УТВЕРЖДЕН\\
RU.17701729.10.03-01 ТЗ 01-1-ЛУ\\
\end{textbf}
\bigskip
\begin{center}
\topskip=0pt
\vspace*{\fill}
\textbf{ПРОГРАММА ДЛЯ ЗАПОМИНАНИЯ ЧИСЛОВЫХ\\
 ДАННЫХ С ИСПОЛЬЗОВАНИЕМ ОСНОВНОЙ\\
 МНЕМОНИЧЕСКОЙ И ДОМИНИКАНСКОЙ СИСТЕМ\\
~\\
~\\
Программа и методика испытаний\\
~\\
RU.17701729.10.03-01 ТЗ 01-1-ЛУ}\\
~\\
Листов \ztotpages\\
\vspace*{\fill}
\end{center}
\begin{center}
\vspace*{\fill}{
  Москва 2022}
\end{center}
\newpage
\begin{center}
\section {Содержание}
\tableofcontents
\end{center}
\newpage
\section {Аннотация}
В данном документе представлена пояснительная записка к программе "{}mnemonic-systems-app.apk"{}, реализующей Android-приложение "{}Программа для запоминания числовых данных с использованием основной мнемонической и Доминиканской систем"{}. Данная программа предназначена для использования на мобильном устройстве при необходимости запоминания больших чисел.\\
~\\
Настоящий документ разработан в соответствии с требованиями:
\begin{itemize}
    \item ГОСТ 19.101-77 Виды программ и программных документов [1];
    \item ГОСТ 19.102-77 Стадии разработки [2];
    \item ГОСТ 19.103-77 Обозначения программ и программных документов [3];
    \item ГОСТ 19.104-78 Основные надписи [4];
    \item ГОСТ 19.105-78 Общие требования к программным документам [5];
    \item ГОСТ 19.106-78 Требования к программным документам, выполненным печатным
способом [6];
    \item ГОСТ 19.404-79 Пояснительная записка. Требования к содержанию и оформлению [7].
Изменения к данному Техническому заданию оформляются согласно ГОСТ 19.603-78 [8],
ГОСТ 19.604-78 [9].
\end{itemize}
Изменения к данному Техническому заданию оформляются согласно ГОСТ 19.603-78 [8],
ГОСТ 19.604-78 [9].
\newpage
\begin{center}
\section {Содержание}
\tableofcontents
\end{center}
\newpage
\section{Введение}
\subsection{Наименование программы}
\subsection{Документ, на основании которого ведётся разработка}
\newpage
\section{Назначение и область применения}
\subsection{Назначение программы}
\subsection{Область применения}
\newpage
\section{Технические характеристики}
\newpage
\section{Ожидаемые технико-экономические показатели}
\subsection{Первоначальная оценка успеха проекта}
Соответствие написанного приложения заявленным требованиям (функциональное тестирование).
\subsection{Последующая оценка}
Оценка приложения принимающей комиссией на защите курсовой работы в институте (ДПИ ФКН НИУ ВШЭ).
\subsection{Конечный параметр оценки проекта}
Число пользователей, установивших приложение в магазине приложений Google Play.\\
\subsection{Предполагаемая потребность}
В ходе обзора конкурентов и общения с потенциальными пользователями было выяснено, что существующие решения не закрывают потребности современного рынка. Поэтому полученное приложение может быть востребовано людьми, имеющими потребность в запоминании больших чисел. Для определения размеров подобной целевой аудитории необходим глубокий социологический анализ, не предусмотренный в рамках данной работы.
\subsection{Ориентировочная экономическая эффективность}
На начальной стадии развития проекта приложение распространяется бесплатно, поэтому не является экономически эффективным. Однако в будущем возможен более подробный расчёт возможной стоимости продукта.
\subsection{Экономические преимущества разработки по сравнению с отечественными и
зарубежными аналогами}
Подробное рассмотрение аналогов приведено в пункте 3.2. Сравнительный анализ показал, что разрабатываемое в рамках курсовой работы приложение является конкурентно способным по сравнению с аналогами.
\begin{table}[H]
\caption{\label{tab:canonsummary}Сравнительный анализ разрабатываемого приложения.}
\begin{center}
\begin{tabular}{|c|c|c|c|c|}
\hline
\textbf{Название} & \textbf{Современный дизайн} & \textbf{Выбор языков} & \textbf{Сохранение чисел} & \textbf{Словарь}\\
\hline
\multicolumn{4}{|c|}{\textbf{Приложения для компьютера}} \\
\hline
010 Memorizer
& $-$
& $-$
& $-$
& $+$ \\
\hline
2Know
& $-$
& $+$
& $-$
& $-$ \\
\hline
\multicolumn{4}{|c|}{\textbf{Web-приложения}} \\
\hline
peoplebyinititals
& $+$
& $-$
& $-$
& $-$ \\
\hline
\multicolumn{4}{|c|}{\textbf{Мобильные приложения}} \\
\hline
Mnemonic major system
& $-$
& $-$
& $+$
& $+$ \\
\hline
Major System: Word Generator
& $+$
& $\pm$
& $-$
& $-$ \\
\hline
A+ Major System
& $+$
& $-$
& $+$
& $-$ \\
\hline
Разрабатываемое приложение
& $+$
& $\pm$
& $+$
& $+$ \\
\hline
\end{tabular}
\end{center}
\end{table} 
\newpage
\addcontentsline{toc}{section}{Список использованных источников}
\section*{Список использованных источников}
\begin{thebibliography}{}
\bibitem{litlink1} ГОСТ 19.001-77. Единая система программной документации. Термины и определения: утвержден и введен в действие Постановлением Государственного комитета стандартов Совета Министров СССР от 20 мая 1977 г. № 1268 срок введения: с 01.01.1980 г. – URL: https://www.swrit.ru/doc/espd/19.001-77.pdf (дата обращения: 27.01.2023). – Текст: электронный.
\end{thebibliography}
\newpage
\begin{center}
\addcontentsline{toc}{section}{Приложения}
\section*{Приложения}
\end{center}
\zz{}\textbf{Приложение 1\\}
Ссылка на репозиторий проекта с исходным кодом и всеми использованными материалами.\\
https://github.com/NikPeg/mnemonic\_systems\_app\\
\zz{}\textbf{Приложение 2\\}
Ссылка на проект интерфейса в сервисе Figma, отражающий примерную структуру будущего приложения.\\
https://www.figma.com/file/jBcJmt0PREwHvBQRowhaHO/Mnemonic-systems?node-id=38\%3A250\&t=\\
Q8JXDdb3HXM9gGPh-1\\
\end{document}