\documentclass[draft]{article}
\usepackage{cmap}
\usepackage[T1,T2A]{fontenc}
\usepackage[utf8]{inputenc}
\usepackage[russian]{babel}
\usepackage[left=2cm,right=2cm,top=2cm,bottom=2cm,bindingoffset=0cm]{geometry}
\usepackage{tikz}
\usepackage{setspace,amsmath}
\usepackage{titlesec}
\usepackage{lipsum}
\usepackage[usestackEOL]{stackengine}
\usepackage{kantlipsum}
\usepackage{graphicx}
\usepackage{caption}
\usepackage{float}
\usepackage{zref-totpages}
\usepackage{fancyhdr}
\pagestyle{fancy}
\fancyhf{} 
\fancyhead[C]{\thepage\\ RU.17701729.10.03-01 ТЗ 01-1}
\renewcommand{\headrulewidth}{0pt}
\captionsetup[table]{justification=centering}
\usetikzlibrary{positioning}
\graphicspath{{pictures/}}
\DeclareGraphicsExtensions{.pdf,.png,.jpg}
\newcommand\zz[1]{\par{\normalsize\strut #1} \hfill\ignorespaces}
\addto\captionsrussian{\def\refname{}}
\newcommand{\subtitle}[1]{%
  \posttitle{%
    \par\end{center}
    \begin{center}\Large#1\end{center}
   }%
}
\newcommand{\subsubtitle}[1]{%
  \preauthor{%
    \begin{center}
    \large #1 \vskip0.5em
    \begin{tabular}[t]{c}
    }%
}
\begin{document}
\thispagestyle{empty}
\begin{center}
\textbf{
ПРАВИТЕЛЬСТВО РОССИЙСКОЙ ФЕДЕРАЦИИ\\
НАЦИОНАЛЬНЫЙ ИССЛЕДОВАТЕЛЬСКИЙ УНИВЕРСИТЕТ\\
«ВЫСШАЯ ШКОЛА ЭКОНОМИКИ»\\
Факультет компьютерных наук\\
Образовательная программа «Программная инженерия»\\
(ВШЭ ФКН ПИ)}\\
\end{center}
\bigskip
\zz{СОГЛАСОВАНО}УТВЕРЖДАЮ
\zz{Доцент департамента}Академический руководитель
\zz{Программной инженерии,}образовательной программы
\zz{ФКН, к.т.н.}«Программная инженерия»
\zz{\noindent\rule{3cm}{0.4pt} К. Ю. Дегтярёв}профессор департамента программной
\zz{«\noindent\rule{1cm}{0.4pt}»\noindent\rule{2cm}{0.4pt}20\noindent\rule{0.5cm}{0.4pt}г.}инженерии, к.т.н.
\zz{~}\noindent\rule{3cm}{0.4pt} В.В. Шилов
\zz{~}«\noindent\rule{1cm}{0.4pt}»\noindent\rule{2cm}{0.4pt}20\noindent\rule{0.5cm}{0.4pt}г.
\begin{center}
\topskip=0pt
\vspace*{\fill}
\textbf{ПРОГРАММА ДЛЯ ЗАПОМИНАНИЯ ЧИСЛОВЫХ\\
 ДАННЫХ С ИСПОЛЬЗОВАНИЕМ ОСНОВНОЙ\\
 МНЕМОНИЧЕСКОЙ И ДОМИНИКАНСКОЙ СИСТЕМ\\
~\\
~\\
Пояснительная записка\\
~\\
ЛИСТ УТВЕРЖДЕНИЯ\\
~\\
RU.17701729.10.03-01 ТЗ 01-1-ЛУ}\\
\vspace*{\fill}
\end{center}
\zz{~}Исполнитель
\zz{~}Студент группы БПИ204
\zz{~}образовательной программы
\zz{~}«Программная инженерия»
\zz{~}Пеганов Никита Сергеевич
\zz{~}\noindent\rule{3cm}{0.4pt} Н. С. Пеганов
\zz{~}«\noindent\rule{1cm}{0.4pt}»\noindent\rule{2cm}{0.4pt}20\noindent\rule{0.5cm}{0.4pt}г.
\begin{center}
\vspace*{\fill}{
  Москва \the\year{}}
\end{center}
\newpage
\clearpage
\pagenumbering{arabic} 
\begin{textbf}\\
УТВЕРЖДЕН\\
RU.17701729.10.03-01 ТЗ 01-1-ЛУ\\
\end{textbf}
\bigskip
\begin{center}
\topskip=0pt
\vspace*{\fill}
\textbf{ПРОГРАММА ДЛЯ ЗАПОМИНАНИЯ ЧИСЛОВЫХ\\
 ДАННЫХ С ИСПОЛЬЗОВАНИЕМ ОСНОВНОЙ\\
 МНЕМОНИЧЕСКОЙ И ДОМИНИКАНСКОЙ СИСТЕМ\\
~\\
~\\
Пояснительная записка\\
~\\
RU.17701729.10.03-01 ТЗ 01-1-ЛУ}\\
~\\
Листов \ztotpages\\
\vspace*{\fill}
\end{center}
\begin{center}
\vspace*{\fill}{
  Москва \the\year{}}
\end{center}
\newpage
\begin{center}
\section {Содержание}
\tableofcontents
\end{center}
\newpage
\section {Аннотация}
В данном документе представлена пояснительная записка к программе "{}mnemonic-systems-app.apk"{}, реализующей Android-приложение "{}Программа для запоминания числовых данных с использованием основной мнемонической и Доминиканской систем"{}. Данная программа предназначена для использования на мобильном устройстве при необходимости запоминания больших чисел.\\
~\\
Настоящий документ разработан в соответствии с требованиями:
\begin{itemize}
    \item ГОСТ 19.101-77 Виды программ и программных документов [1];
    \item ГОСТ 19.102-77 Стадии разработки [2];
    \item ГОСТ 19.103-77 Обозначения программ и программных документов [3];
    \item ГОСТ 19.104-78 Основные надписи [4];
    \item ГОСТ 19.105-78 Общие требования к программным документам [5];
    \item ГОСТ 19.106-78 Требования к программным документам, выполненным печатным
способом [6];
    \item ГОСТ 19.404-79 Пояснительная записка. Требования к содержанию и оформлению [7].
Изменения к данному Техническому заданию оформляются согласно ГОСТ 19.603-78 [8],
ГОСТ 19.604-78 [9].
\end{itemize}
Изменения к данному Техническому заданию оформляются согласно ГОСТ 19.603-78 [8],
ГОСТ 19.604-78 [9].
\newpage
\section{Введение}
\subsection{Наименование программы на русском языке}
Программа для запоминания числовых данных с использованием основной мнемонической и Доминиканской систем.
\subsection{Наименование программы на английском языке}
A program for storing numerical data using the basic mnemonic and Dominican systems.
\subsection{Документ, на основании которого ведётся разработка}
Программа разработана в рамках выполнения курсового проекта — "{}Программа для запоминания числовых данных с использованием основной мнемонической и Доминиканской систем"{}, в соответствии с учебным планом 3 курса бакалавриата направления 09.03.04
«Программная инженерия»\cite{litlink2}.\\
~\\
Основание для разработки — учебный план подготовки бакалавров по направлению
09.03.04 «Программная инженерия»\cite{litlink3} и утвержденная академическим руководителем
программы тема курсового проекта.
\newpage
\section{Назначение и область применения}
\subsection{Целевая аудитория продукта}
Люди, которым требуется запоминать большие числа на постоянной основе. Также приложением могут пользоваться люди, которые хотят улучшить свою память. Достаточно распространенными примерами чисел для запоминания являются номера клиента банка, номера банковских карт, номера страховых свидетельств, дни рождения, экстренные номера телефонов, телефоны знакомых, важные в программировании числа, исторические даты, слова иностранных языков, имена и лица. В связи со спецификой предоставляемой в приложении информации, ограничение на возраст пользователя — от 6 лет.
\subsection{Общение с потенциальными пользователями}
Для разработки функционала приложения был проведён CustDev (исследование потребностей клиента с помощью проведения специальных интервью) с несколькими потенциальными пользователями. Приведем краткое содержание их user stories.\\
~\\
\textbf{Интервьюируемый 1:} Существует потребность в запоминании номеров банковских карт, так как часто пользуется ими в интернете. Это является проблемой, так как он не всегда носит с собой кошелек с карточками. Про Доминиканскую и основную мнемоническую систему пользователь не знает. Пользовался приложениями для запоминания такими, как QuizLet, Lingvist, Anki для изучения иностранных языков. Не пробовал приложения для запоминания больших чисел. Пользовался бы подобным приложением, если бы оно поддерживало русский язык.\\
~\\
\textbf{Интервьюируемый 2:} Хотел знать, можно ли использовать в качестве средства для запоминания слов и фраз Доминиканскую и основную мнемоническую систему. Возможно ли это, если не пользуется мобильными устройствами, в основном пользуется компьютером. Мнемоническая система: не знает подробностей. Слышал о ней, но никогда не использовал. Приложений для запоминания не установлено.\\
~\\
\textbf{Интервьюируемый 3:} Пользовался приложением QuizLet для запоминания большого количества цифр (было необходимо на работе). После завершения курса, пользователь начинает забывать цифры. Недостаток — приложение не помогает запоминать большие числа. Читал про Доминиканскую систему, но не пробовал ее применить, так как нужно было потратить время на придумывание ассоциаций.\\
~\\
\textbf{Интервьюируемый 4:} Часто пользуется банковской картой, но не помнит ее номер. Он не пользуется приложением для запоминания чисел или слов, так как ему это не нужно. Был бы рад запомнить свой номер паспорта для более быстрого заполнения документов, но проблемы в этом не видит, так как всегда носит паспорт с собой. Не знает про системы запоминания чисел.\\
~\\
\textbf{Интервьюируемый 5:} Несколько лет назад изучил основную мнемоническую систему для того, чтобы выучить физические константы для более быстрого решения задач. Попробовал приложение 010 Memorizer, но было сложно пользоваться им на компьютере, а также не было возможности использовать русский язык. В дальнейшем использовал бумажные карточки для букв-ассоциаций. Сейчас не пользуется системами запоминания, так как пропала необходимость.\\
~\\
\textbf{Интервьюируемый 6:} Есть потребность запоминания номера паспорта, банковской карты и математических констант. Для этого она рассматривала число, находила у него какое-нибудь особенное свойство и запоминала его. То есть привязывала цифры к шутке, к смешной картинке, к какому-то образу. Про Доминиканскую и основную мнемоническую системы не слышала. Обладает хорошей зрительной памятью.\\
~\\
\textbf{Выводы:}
\begin{enumerate}
\item Большинство опрошенных пользователей не знают или знают мало о Доминиканской и основной мнемонической системах. Стоит добавить в приложение информацию о них, чтобы пользователь мог ее правильно применить.
\item Для части пользователей важно, чтобы приложение позволяло использовать русский язык. Существующие приложения не предоставляют такую возможность.
\item У пользователей существуют общие потребности в запоминании больших чисел: номера паспорта, банковских карт и прочие. Можно сделать в приложении функционал, ориентированный на популярные виды чисел.
\item Для большинства пользователей самым удобным было бы приложение для смартфона. Оно позволяет запоминать ассоциации и числа в дороге, во время отдыха. Компьютер же большую часть времени занят рабочими задачами.
\item В существующих приложениях мало возможностей для создания ассоциаций с нужными числами. Например, можно реализовать возможность делать наброски в приложении, а также прикреплять фотографии.
\end{enumerate}
\subsection{Актуальность проблемы}
Проблема запоминания больших чисел — вечная проблема, так как память человека не менялась значительно на протяжении истории. Поэтому приложение, позволяющее быстро запоминать числовые данные может быть полезно до тех пор, пока будут существовать мобильные телефоны.
\subsection{Функциональное назначение}
Программа представляет из себя удобное приложение для хранения и запоминания чисел, а также изучения основной мнемонической и Доминиканской систем. Приложение не ограничивает пользователя в количестве сохраненных им чисел, ограничивающим фактором является только память телефона.\\
Приложение делится на 5 основных разделов:
\begin{itemize}
\item Запоминаемые числа;
\item Информация об основной мнемонической системе;
\item Справка;
\item Информация о Доминиканской системе;
\item Личный кабинет.
\end{itemize}
Более подробное описание элементов программы представлено в следующем разделе.
\subsection{Эксплуатационное назначение}
Приложение предназначено для пользователей мобильных устройств от 6 лет, сталкивающихся с проблемой запоминания больших чисел. Доступ в интернет не является необходимым для работы программы.
\subsection{Область применения программы}
Программа предназначена для запоминания числовых данных. Она может быть использована как в личных целях, например, для запоминания дней рождений, так и для профессиональных, например, в бухгалтерии. Другие профессиональные сферы в которых может применяться программа:
\begin{itemize}
\item медицина, где врачам и медсестрам необходимо запоминать множество медицинских данных, кодов и номеров пациентов;
\item бухгалтерия, где бухгалтерам необходимо запоминать множество номеров счетов, кодов и других финансовых данных;
\item учеба, где студентам необходимо запоминать большое количество информации, такой как формулы, исторические даты, географические данные и т.д.;
\item наука, где ученым необходимо запоминать множество констант, чисел и других научных данных;
\item бизнес, где менеджерам и предпринимателям необходимо запоминать множество номеров телефонов, адресов, кодов и других контактных данных;
\item другие сферы жизни.
\end{itemize}
\newpage
\section{Технические характеристики}
\subsection{Описание использованных мнемонических систем}
Основная мнемоническая и Доминиканская системы не являются широко распространенными в русскоязычной среде. Цель данного приложения — позволить русскоговорящему пользователю применять обе мнемонические системы с помощью смартфона.\\
~\\
Мнемоническая система - это метод запоминания информации с помощью ассоциаций и образов, которые легко запоминаются. Эта система основывается на использовании изображений, звуков и других сенсорных входов для создания связей между новой информацией и уже имеющимися знаниями. Она может быть использована для запоминания любых видов информации, включая числа, слова, факты и другое. Основная мнемоническая система является одним из наиболее эффективных способов запоминания информации и может быть использована любым человеком в любом возрасте\cite{litlink7}.\\
~\\
Основная мнемоническая система (также называемая фонетической системой счисления, фонетической мнемонической системой или мнемонической системой Херигона) — это мнемоническая техника, используемая для помощи в запоминании чисел. Система работает путем преобразования чисел в согласные, а затем в слова путем добавления гласных. Система работает по принципу, согласно которому слова и образы запоминаются легче, чем цифры\cite{litlink8}.\\
~\\
Доминиканская система — это мнемоническая система, используемая для запоминания последовательностей цифр, аналогичная основной мнемонической системе. Он был изобретен и использовался на соревнованиях восьмикратным чемпионом мира по запоминанию Домиником О'Брайеном. При использовании системы Доминика каждая пара цифр сначала ассоциируется с человеком, чьи инициалы начинаются на соответствующие буквы. Доминик считает, что истории и изображения, созданные с использованием людей, легче запоминаются. Это кодирование выполняется заранее, и люди используются повторно, поскольку это может занять довольно много времени.\cite{litlink9}\\
~\\
Основное различие между системой Доминика и мажорной системой заключается в присвоении звуков и букв цифрам. Система Доминика - это система сокращений, основанная на буквах, где буквы составляют инициалы чьего-либо имени, в то время как основная система обычно используется как фонетическая система согласных для обозначения объектов, животных, людей или даже слов.
\subsection{Алгоритм функционирования программы}
Для решения поставленной в техническом задании задачи была использована структура данных Бор\cite{litlink5} и алгоритм поиска по нему. Бор — структура данных для хранения набора строк, представляющая из себя подвешенное дерево с символами на рёбрах. Строки получаются последовательной записью всех символов, хранящихся на рёбрах между корнем бора и терминальной вершиной. Размер бора линейно зависит от суммы длин всех строк, а поиск в бору занимает время, пропорциональное длине образца.\\
~\\
\textbf{Алгоритм построения дерева в структуре данных Бор}\\
\begin{itemize}
\item Шаг 1. Создадим дерево из одной вершины (в нашем случае корня).
\item Шаг 2. Добавляем элементы в дерево.
\subitem Добавляем шаблоны $P_i$ один за другим. Следуем из корня по рёбрам, отмеченным буквами из $P_i$, пока возможно.
\subitem Если $P_i$ заканчивается в $v$, сохраняем идентификатор $P_i$ (например, $i$) в $v$ и отмечаем вершину $v$ как терминальную.
\subitem Если ребра, отмеченного очередной буквой $P_i$ нет, то создаем новое ребро и вершину для символа строки $P_i$.
\end{itemize}
Построение занимает, очевидно, $O(|P1|+…+|Pk|)=O(n)$ времени, так как поиск буквы, по которой нужно переходить, происходит за $O(1)$.\\
~\\
\textbf{Алгоритм поиска строки в структуре данных Бор}\\
При решении этой задачи, обход бора совершается из его корня по рёбрам, отмеченным символами строки S
, пока возможно. Если с последним символом S мы приходим в терминальную вершину, то S — слово из словаря. Если в какой-то момент ребра, отмеченного нужным символом, не находится, то строки S в словаре нет. Ясно, что это занимает $O(|S|)$ времени. Таким образом, бор — это эффективный способ хранить словарь и искать в нем слова\cite{litlink5}.\\
~\\
В данной работе алгоритм Бор решает задачу поиска слов, соответствующим введённым пользователем чисел в основной мнемонической и в Доминиканской системах. Например, числу 314 в основной мнемонической системе будет соответвовать слово "{}метеор"{}. Оно окажется одним из слов в узле Бора, в которое ведут рёбра 3, 1 и 4.\\
~\\
Наиболее длинное слово в русском языке — "{}тетрагидропиранилциклопентилтетрагидропиридопиридиновые"{}, его длина — 55 букв\cite{litlink6}. Это значит, что поиск данного слова в структуре Бора займёт время, пропорциональное длине строки — $O(55)=O(1)$. Поэтому выбранный алгоритм находит все возможные слова за незначительное время, что делает его подходящим данной задаче.
\subsection{Интерфейс программы}
Приложение состоит из 11 экранов (результат предварительного проектирования по состоянию на конец марта 2023 года):
\begin{itemize}
\item \textbf{Запоминаемые числа}\\
На данном экране представлен список запоминаемых чисел в виде "{}плиток"{}\ — выделяющихся на фоне основного приложения прямоугольных элементов. На каждой из них расположена картинка, выбранная пользователем, если она есть. Под картинкой — его название и само число. В случае, если число не помещается в пределах плитки, оно сокращается до первых нескольких символов. В правой нижней части экрана присутствует кнопка "{}Добавить"{}\ , переводящая пользователя на следующий экран.
\item \textbf{Добавление числа}\\
При добавлении числа открывается экран с двумя полями ввода — описанием числа и самим числом. Также присутствует возможность выбора изображения из памяти телефона. На этой же странице присутствует выбор мнемонической системы среди вариантов: основная мнемоническая, Доминиканская, без системы.
\item \textbf{Поиск слов для запоминания}\\
В верхней части экрана указано пояснение: "{}Выберете цифры для поиска слов:"{}. Под ним отражены цифры указанного пользователем числа в виде кнопок. При нажатии поочередно каждой из этих кнопок, под ними появляется список слов, соответствующих выбранным цифрам в указанной мнемонической системе. При нажатие любого из этих слов, они добавляются в поле ввода в нижней части экрана. Под этим полем ввода присутствует кнопка "{}Сохранить число"{}.
\item \textbf{Редактирование числа}\\
При нажатии на любое из чисел в экране "{}Запоминаемые числа"{}\ открывается экран редактирования числа. На нем есть возможность изменить описание, число, фразу для запоминания в соответствующих текстовых полях. Также возможно выбрать другое фото. В нижней части экрана — кнопка "{}Сохранить"{}.
\item \textbf{Карточки с числами в основной мнемонической системе}\\
Этот экран нужен для того, чтобы позволить пользователю выучить соответствие между цифрой и буквой в основной мнемонической системе. Для этого на главном экране расположена прямоугольная карточка с изображением случайной цифры. При нажатии на нее она переворачивается, показывая соответствующую букву. При ее листании открывается следующая карточка.
\item \textbf{Справка с информацией об обеих системах}\\
На данном экране представлена информация об основной мнемонической и Доминиканской системах в виде текста, изображений, а также таблиц.
\item \textbf{Карточки с числами в Доминиканской системе}\\
Экран нужен для того, чтобы позволить пользователю выучить соответствие между цифрой и буквой в Доминиканской системе. Для этого на главном экране расположена прямоугольная карточка с изображением случайной цифры. При нажатии на нее она переворачивается, показывая соответствующую букву. При ее листании открывается следующая карточка.
\item \textbf{Панель настроек}\\
Панель настроек отображает список из всех главных экранов, отраженных в нижнем меню, с их описаниями. Также присутствует "{}Выбор языка"{}\ и "{}Словарь"{}. Справа от каждого из пунктов настроек приведен соответствующий ему логотип.
\item \textbf{Список слов в словаре}\\
В верхней части экрана расположен поиск слов, под ним — список всех слов словаря в алфавитном порядке. В правой нижней части экрана присутствует кнопка "{}Добавить"{}, позволяющая добавить слово в словарь.
\item \textbf{Добавление слова в словарь}\\
Всплывающее окно с единственным полем ввода "{}введите слово"{}\ и кнопкой "{}Добавить"{}.
\item \textbf{Выбор языка}\\
Всплывающее окно с выпадающим списков языков и кнопкой "{}Выбрать язык"{}. В данный момент предоставляется выбор из двух языков: английского и русского.
\end{itemize}
Примерный вид описанных экранов может быть увиден в прототипе интерфейса, созданном в приложении Figma. См. приложение №2.\\
~\\
\textbf{Шаблон страницы}\\
Экран всегда вертикальный, разворот запрещен.\\
~\\
\textbf{Шапка}\\
Шапка страницы меняется в зависимости от текущей страницы. На ней появляется название страницы, а также иконка поиска, если это необходимо на текущей странице.\\
Для всех внутренних страниц (не обозначенных в нижнем меню) должна быть кнопка
Вернуться.\\
~\\
\textbf{Подвал}\\
Основное меню располагается в нижней части экрана. Разделы обозначаются иконками:
\begin{itemize}
\item Раздел "{}Запоминаемые числа"{} отражён с помощью иконки "{}123"{}, символизирующей числа.
\item Раздел "{}Карточки с числами в основной мнемонической системе"{}\ отражён иконкой "{}MMS"{}, означающей сокращение от названия "{}mnemonic major system"{}.
\item Раздел "{}Справка с информацией"{}\ символизируется вопросительным знаком.
\item Раздел "{}Карточки с числами в Доминиканской системе"{}\ отражаются икокной "{}DS"{}. Это сокращение от "{}Dominic system"{}.
\item Раздел настроек отражается классической шестеренкой, символизирующей настройки.
\end{itemize}
\textbf{Требование к интерфейсу}\\
Интерфейс должен быть оформлен в соответствии с дизайн-системой Material Design.
\subsection{Выбор технических и программных средств}
Программа может быть запущена на мобильном телефоне или планшете с операционной системой Android версии 7.0 и выше. Требования к составу и параметрам технического средства соответствуют требованиям данной операционной системы. Дополнительных ограничений не накладывается.\\
~\\
Разработка велась в Android Studio. Android Studio — интегрированная среда разработки для работы с платформой Android, анонсированная 16 мая 2013 года на конференции Google I/O \cite{litlink4}.
\newpage
\section{Ожидаемые технико-экономические показатели}
\subsection{Первоначальная оценка успеха проекта}
Соответствие написанного приложения заявленным требованиям (функциональное тестирование).
\subsection{Последующая оценка}
Оценка приложения принимающей комиссией на защите курсовой работы в институте (ДПИ ФКН НИУ ВШЭ).
\subsection{Конечный параметр оценки проекта}
Число пользователей, установивших приложение в магазине приложений Google Play.\\
\subsection{Предполагаемая потребность}
В ходе обзора конкурентов и общения с потенциальными пользователями было выяснено, что существующие решения не закрывают потребности современного рынка. Поэтому полученное приложение может быть востребовано людьми, имеющими потребность в запоминании больших чисел. Для определения размеров подобной целевой аудитории необходим глубокий социологический анализ, не предусмотренный в рамках данной работы.
\subsection{Ориентировочная экономическая эффективность}
На начальной стадии развития проекта приложение распространяется бесплатно, поэтому не является экономически эффективным. Однако в будущем возможен более подробный расчёт возможной стоимости продукта.
\subsection{Экономические преимущества разработки по сравнению с отечественными и
зарубежными аналогами}
Подробное рассмотрение аналогов приведено в пункте 3.2. Сравнительный анализ показал, что разрабатываемое в рамках курсовой работы приложение является конкурентно способным по сравнению с аналогами.
\begin{table}[H]
\caption{\label{tab:canonsummary}Сравнительный анализ разрабатываемого приложения.}
\begin{center}
\begin{tabular}{|c|c|c|c|c|}
\hline
\textbf{Название} & \textbf{Современный дизайн} & \textbf{Выбор языков} & \textbf{Сохранение чисел} & \textbf{Словарь}\\
\hline
\multicolumn{4}{|c|}{\textbf{Приложения для компьютера}} \\
\hline
010 Memorizer
& $-$
& $-$
& $-$
& $+$ \\
\hline
2Know
& $-$
& $+$
& $-$
& $-$ \\
\hline
\multicolumn{4}{|c|}{\textbf{Web-приложения}} \\
\hline
peoplebyinititals
& $+$
& $-$
& $-$
& $-$ \\
\hline
\multicolumn{4}{|c|}{\textbf{Мобильные приложения}} \\
\hline
Mnemonic major system
& $-$
& $-$
& $+$
& $+$ \\
\hline
Major System: Word Generator
& $+$
& $\pm$
& $-$
& $-$ \\
\hline
A+ Major System
& $+$
& $-$
& $+$
& $-$ \\
\hline
Разрабатываемое приложение
& $+$
& $\pm$
& $+$
& $+$ \\
\hline
\end{tabular}
\end{center}
\end{table} 
\newpage
\addcontentsline{toc}{section}{Список использованных источников}
\section*{Список использованных источников}
\begin{thebibliography}{}
\bibitem{litlink1} ГОСТ 19.001-77. Единая система программной документации. Термины и определения: утвержден и введен в действие Постановлением Государственного комитета стандартов Совета Министров СССР от 20 мая 1977 г. № 1268 срок введения: с 01.01.1980 г. – URL: https://www.swrit.ru/doc/espd/19.001-77.pdf (дата обращения: 27.01.2023). – Текст: электронный.
\bibitem{litlink2} \textit{Учебный офис ФКН ПИ} (2023) СПРАВОЧНИК УЧЕБНОГО ПРОЦЕССА НИУ ВШЭ. Курсовая работа // Сайт hse.ru (https://www.hse.ru/studyspravka/kursovrab/) Просмотрено: 31.03.2023.
\bibitem{litlink3} \textit{Пак Татьяна Альбертовна} (2023) Бакалаврская программа «Программная инженерия» // Сайт hse.ru (https://www.hse.ru/ba/se/passport) Просмотрено: 31.03.2023.
\bibitem{litlink4} \textit{AndroidDev} (2022) Meet Android Studio // Сайт developer.android.com (https://developer.android.com/studio/intro) Просмотрено: 31.03.2023.
\bibitem{litlink5} \textit{студенты ИТМО} (4 сентября 2022) Бор // Сайт neerc.ifmo.ru (https://neerc.ifmo.ru/wiki/index.php?title=\%D0\%91\%D0\%BE\%D1\%80) Просмотрено: 31.03.2023.
\bibitem{litlink6} \textit{Посольство Российской Федерации в Кыргызстане} (13 сентября 13) Самые длинные слова в русском языке // Сайт rusinkg.ru (http://www.rusinkg.ru/russkij-yazyk/article/43-velikij-i-moguchij/194-samye-dlinnye-slova-russkogo-yazyka) Просмотрено: 31.03.2023.
\bibitem{litlink7} \textit{Jonathan Ströbele} (2013) Major System database // Сайт major-system.info (https://major-system.info/en/) Просмотрено: 07.04.2023.
\bibitem{litlink8} \textit{Hale-Evans, Ron} (2006) Mind performance hacks // Сайт archive.org (https://archive.org/details/mindperformanceh00hale/page/14/mode/2up) Просмотрено: 07.04.2023.
\bibitem{litlink9} \textit{Hale-Evans, Ron} (2006) Mind performance hacks // Сайт archive.org (https://archive.org/details/mindperformanceh00hale/page/22/mode/2up) Просмотрено: 07.04.2023.
\end{thebibliography}
\newpage
\begin{center}
\addcontentsline{toc}{section}{Приложения}
\section*{Приложения}
\end{center}
\zz{}\textbf{Приложение 1\\}
Ссылка на репозиторий проекта с исходным кодом и всеми использованными материалами.\\
https://github.com/NikPeg/mnemonic\_systems\_app\\
\zz{}\textbf{Приложение 2\\}
Ссылка на проект интерфейса в сервисе Figma, отражающий примерную структуру будущего приложения.\\
https://www.figma.com/file/jBcJmt0PREwHvBQRowhaHO/Mnemonic-systems?node-id=38\%3A250\&t=\\
Q8JXDdb3HXM9gGPh-1\\
\end{document}